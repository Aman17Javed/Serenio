
\documentclass[a4paper,12pt]{article}
\usepackage[utf8]{inputenc}
\usepackage{geometry}
\geometry{a4paper, margin=1in}
\usepackage{times}
\usepackage{fancyhdr}
\pagestyle{fancy}
\fancyhf{}
\fancyhead[L]{Serenio Mental Health}
\fancyhead[R]{Session Report}
\fancyfoot[C]{\thepage}
\usepackage{hyperref}

\begin{document}

\begin{center}
{\Huge \textbf{Serenio Mental Health Session Report}} \\
\vspace{0.5cm}
{\large Generated on: 7/29/2025, 1:02:46 AM} \\
\vspace{1cm}
\hrule
\end{center}

\section*{Patient Information}
\begin{itemize}
  \item \textbf{Patient Name:} AMAN
  \item \textbf{Patient ID:} 6886030d828acd6f462765ae
  \item \textbf{Session ID:} 543d8b7c-a120-4368-b5e6-8f7d92a66302
  \item \textbf{Date & Time:} 7/29/2025, 1:02:46 AM
\end{itemize}

\section*{Report Summary}
**Conversation Summary:** The user expressed feelings of distress due to exam-related stress and overwhelming emotions, including anger. The conversation focused on acknowledging these feelings and exploring coping strategies. \\  \\ **Emotional Analysis:** The dominant tone was neutral, with shifts towards stress and frustration as the user articulated their feelings about exams and anger. \\  \\ **Risk Assessment:** Moderate levels of stress and anxiety were indicated; no suicidal ideation or severe withdrawal was noted. \\  \\ **Recommendations:** Focus on stress management techniques, including deep breathing, physical activity, and expressing emotions constructively. Encouraging the user to utilize a study schedule and seek support from trusted individuals is also advisable.

\vspace{1cm}
\hrule
\end{document}
    